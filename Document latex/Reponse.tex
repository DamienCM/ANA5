

%----------------------------------------------------------------------------------------
%	PACKAGES AND OTHER DOCUMENT CONFIGURATIONS
%----------------------------------------------------------------------------------------

\documentclass[12pt]{report}
\usepackage[french]{babel}
\usepackage[utf8]{inputenc} % Required for inputting international characters
\usepackage[T1]{fontenc} % Output font encoding for international characters
\usepackage{fancyhdr}
\usepackage{hyperref}
\usepackage{amsmath}
\usepackage{amsfonts}
\usepackage{amssymb}
\usepackage{tikz,pgfplots,pgf}
\pgfplotsset{compat=1.16}
\usepackage{mathrsfs}
\usetikzlibrary{arrows}




\addto\captionsfrench{\renewcommand{\chaptername}{Cas}}

\begin{document}

%----------------------------------------------------------------------------------------
%	TITLE PAGE
%----------------------------------------------------------------------------------------

\begin{titlepage} % Suppresses displaying the page number on the title page and the subsequent page counts as page 1
	\newcommand{\HRule}{\rule{\linewidth}{0.5mm}} % Defines a new command for horizontal lines, change thickness here
	
	\center % Centre everything on the page
	
	%------------------------------------------------
	%	Headings
	%------------------------------------------------
	
	\textsc{\LARGE INSA de Strasbourg}\\[1.2cm] % Main heading such as the name of your university/college
	
	\textsc{\Large Projet Mathématique}\\[0.5cm] % Major heading such as course name
	
	
	%------------------------------------------------
	%	Title
	%------------------------------------------------
	
	\HRule\\[0.4cm]
	
	{\huge\bfseries Problème isopérimétrique}\\[0.4cm] % Title of your document
	
	\HRule\\[1.5cm]
	
	%------------------------------------------------
	%	Author(s)
	%------------------------------------------------
	
	\begin{minipage}{0.4\textwidth}
		\begin{flushleft}
			\large
			\textit{Auteurs}\\
			\textsc{Abrini} Mouad\\ % Your name
			\textsc{Cartier Millon} Damien % Your name
		\end{flushleft}
	\end{minipage}
	~
	\begin{minipage}{0.4\textwidth}
		\begin{flushright}
			\large
			\textit{Encadrant}\\
			M. Jean Romain \textsc{Heu} % Supervisor's name
		\end{flushright}
	\end{minipage}
	
	% If you don't want a supervisor, uncomment the two lines below and comment the code above
	%{\large\textit{Author}}\\
	%John \textsc{Smith} % Your name
	
	%------------------------------------------------
	%	Date
	%------------------------------------------------
	
	\vfill\vfill\vfill % Position the date 3/4 down the remaining page
	
	{\large\today} % Date, change the \today to a set date if you want to be precise
	
	%------------------------------------------------
	%	Logo
	%------------------------------------------------
	
	%\vfill\vfill
	%\includegraphics[width=0.2\textwidth]{placeholder.jpg}\\[1cm] % Include a department/university logo - this will require the graphicx package
	 
	%----------------------------------------------------------------------------------------
	
	\vfill % Push the date up 1/4 of the remaining page
	
\end{titlepage}

%----------------------------------------------------------------------------------------
\tableofcontents
\part{Fonctions de plusieurs variables}

\chapter{Cas du triangle}
\section{Question 1}

Comme indiqué sur la question, on peut utiliser la formule de Héron pour calculer l'aire d'un triangle.
Pour ce faire, il suffit d'avoir le périmètre du triangle.\\
\indent Soit $f:R^{3}\longrightarrow$ R une fonction de classe $C^{1}$ , $g:R^{3}\longrightarrow$R une fonction de classe
$C^{1}$ et soit $ G = \lbrace (x_{1},...,x_{n})\vert g(x_{1},...,x_{n})=0\rbrace.$\\
\indent Soit P le périmètre et A l'aire d'un triangle dont les côtés ont pour mesures a, b et c. Posons alors\\
\begin{equation}\label{eq:11}
   g_{1}(a,b,c) = a + b + c - P
\end{equation}
La fonction $g_{1}$ représente la première contrainte qui est unique dans notre cas.\\
Posons $s = \frac{P}{2}$ le demi-périmètre qui sera fixé.\\ \\
La formule de Héron nous affirme que :
\begin{equation}
   f(a,b,c) = A(a,b,c) = \sqrt[2]{s(s-a)(s-b)(s-c)}
\end{equation}
Nous cherchons à maximiser cette fonction f qui est associée à l'aire de notre triangle.\\


\clearpage
\section{Question 2}
Nous devons ainsi calculer le gradient de g et f.

\begin{align*}
\overrightarrow{\nabla}f(a,b,c) = 
\left(\begin{matrix}
\frac{\partial f(a,b,c)}{\partial a} \\ \\
\frac{\partial f(a,b,c)}{\partial b} \\ \\
\frac{\partial f(a,b,c)}{\partial c} \\ \\
\end{matrix}\right),\quad          &  \overrightarrow{\nabla}g_{1}(a,b,c) = 
\left(\begin{matrix}
\frac{\partial g_{1}(a,b,c)}{\partial a} \\ \\
\frac{\partial g_{1}(a,b,c)}{\partial b} \\ \\
\frac{\partial g_{1}(a,b,c)}{\partial c} \\ \\
\end{matrix}\right)      
\end{align*}
\indent Ce qui nous donne (en utilisant une fonction Python) :
\begin{align*}
\overrightarrow{\nabla}f(a,b,c) = 
\left(\begin{matrix}
- \frac{\sqrt{s \left(- a + s\right) \left(- b + s\right) \left(- c + s\right)}}{2 \left(- a + s\right)}\\ \\
- \frac{\sqrt{s \left(- a + s\right) \left(- b + s\right) \left(- c + s\right)}}{2 \left(- b + s\right)}\\ \\
- \frac{\sqrt{s \left(- a + s\right) \left(- b + s\right) \left(- c + s\right)}}{2 \left(- c + s\right)} \\ \\
\end{matrix}\right),\quad          &  \overrightarrow{\nabla}g_{1}(a,b,c) = 
\left(\begin{matrix}
1 \\ 
1 \\ 
1 \\ 
\end{matrix}\right)      
\end{align*}
\section{Question 3}
\indent L'équation \eqref{eq:11} nous donne que $g_{1}(a,b,c) = 0$.\ Ce qui veut dire que le triplet (a,b,c) appartient à G : Le théorème des extremas liées s'applique.\\
On a donc 
\begin{equation}\label{eq:13}
   \exists\lambda\in R^{+*} \quad \quad \overrightarrow{\nabla}f = \lambda \overrightarrow{\nabla}g_{1}
\end{equation}
En utilisant \eqref{eq:13}, nous obtenons le système suivant :
\begin{equation}
\begin{cases} - \frac{\sqrt{s \left(- a + s\right) \left(- b + s\right) \left(- c + s\right)}}{2 \left(- a + s\right)} = \lambda \quad (1)\\
- \frac{\sqrt{s \left(- a + s\right) \left(- b + s\right) \left(- c + s\right)}}{2 \left(- b + s\right)} = \lambda \quad (2) \\
- \frac{\sqrt{s \left(- a + s\right) \left(- b + s\right) \left(- c + s\right)}}{2 \left(- c + s\right)} = \lambda 
\quad (3) \end{cases}
\end{equation}
On a alors directement en utilisant l'équation (1) et (2) $\frac{s(s-b)(s-c)}{s-a} = \frac{s(s-a)(s-c)}{s-b}$
Qui se simplifie en $(s-a)^{2}=(s-b)^{2}$.
Or on sait que le demi-périmètre est toujours plus grand que chaque coté du triangle. Donc a = b.
on faisant de même avec (2) et (3), on retrouve finalement que a = b = c.\\
Ainsi, en utilisant \eqref{eq:11}, \ on obtient que 
\begin{equation}
   \boxed{a = b = c = \frac{P}{3}}
\end{equation}
Ce qui correspond à un triangle \textbf{équilatéral}.
\chapter{Cas d'un quadrilatère}
\section{Question 1}
\indent Pour simplifier les calculs, on va utiliser le \textbf{théorème isopérimétrique} pour un quadrilatère, qui affirme que 
pour l'aire d'un quadrilatère à périmètre fixé soit maximale, il faut qu'il soit régulier. Nous éviterons ainsi le cas des quadrilatères concaves et croisés.\\
Soit Q un quadrilatère régulier comme représenté sur la figure 


\begin{center}
\begin{tikzpicture}[scale=1.25]%,cap=round,>=latex]
\coordinate [label=left:$A$] (A) at (-1.5cm,-1.cm);
\coordinate [label=right:$B$] (B) at (3cm,-1.0cm);
\coordinate [label=above:$C$] (C) at (1.5cm,1.0cm);
\coordinate [label=above:$D$] (D) at (0cm,1.5cm);
\draw (A) -- node[below] {$x$} (B) -- node[right] {$y$}(C) -- node[above] {$z$} (D)--node[left] {$w$}(A);
\draw (A)--node[right]{$d$}(C);
\end{tikzpicture}
\end{center}
On a ainsi $P = x+y+z+w$.\\
\indent Soit $f:R^{4}\longrightarrow$ R une fonction de classe $C^{1}$ , $g:R^{4}\longrightarrow$R une fonction de classe
$C^{1}$ et soit $ G = \lbrace (x_{1},...,x_{n})\vert g(x_{1},...,x_{n})=0\rbrace.$\\
\indent 
Posons alors
\begin{equation}\label{eq:21}
 g_{1}(x,y,z,w,d)=x+y+z+w-P
\end{equation}
$g_{1}$ représente notre première contrainte.\\
Il est clairement remarquable que ce quadrilatère peut se diviser en deux triangle ABC et ADC.\ Ce qui nous permet pour avoir l'aire totale de Q qui est la somme des aires des deux triangles.\ On notera $s$ le demi-périmètre de ABC et $t$ celui de ADC.\\
On aura ainsi en utilisant la formule d'Héron:
\begin{equation}
    f(x,y,z,w,d)=\sqrt{s(s-x)(s-w)(s-d)}+\sqrt{t(t-y)(t-x)(t-d)}
\end{equation}
On cherche à maximiser cette fonction qui est associée à l'aire de notre quadrilatère.\\
Calculons alors le gradient de f et de $g_{1}$.\ On obtient\\
\begin{align*}
\overrightarrow{\nabla}f(x,y,z,w,d) = 
\left(\begin{matrix}
- \frac{\sqrt{s \left(- d + s\right) \left(s - w\right) \left(s - x\right)}}{2 \left(s - x\right)}\\ \\
- \frac{\sqrt{t \left(- d + t\right) \left(t - y\right) \left(t - z\right)}}{2 \left(t - y\right)}\\ \\
- \frac{\sqrt{t \left(- d + t\right) \left(t - y\right) \left(t - z\right)}}{2 \left(t - z\right)} \\ \\
- \frac{\sqrt{s \left(- d + s\right) \left(s - w\right) \left(s - x\right)}}{2 \left(s - w\right)}\\ \\
- \frac{\sqrt{s \left(- d + s\right) \left(s - w\right) \left(s - x\right)}}{2 \left(- d + s\right)} - \frac{\sqrt{t \left(- d + t\right) \left(t - y\right) \left(t - z\right)}}{2 \left(- d + t\right)}
\end{matrix}\right),\quad          &  \overrightarrow{\nabla}g_{1}(a,b,c) = 
\left(\begin{matrix}
1 \\ 
1 \\ 
1 \\ 
1\\
0
\end{matrix}\right)      
\end{align*}
l'équation \eqref{eq:21} nous affirme que $ g_{1}(x,y,z,w,d)=0$.\ Donc le théorème des extremas liées s'applique. On a ainsi:
\begin{equation}\label{eq:23}
   \exists\lambda\in R^{+*} \quad \quad \overrightarrow{\nabla}f = \lambda \overrightarrow{\nabla}g_{1}
\end{equation}
En utilisant \eqref{eq:23}, nous obtenons le système suivant :
\begin{equation}
\begin{cases} - \frac{\sqrt{s \left(- d + s\right) \left(s - w\right) \left(s - x\right)}}{2 \left(s - x\right)} = \lambda \\
- \frac{\sqrt{t \left(- d + t\right) \left(t - y\right) \left(t - z\right)}}{2 \left(t - y\right)}\ = \lambda  \\
- \frac{\sqrt{t \left(- d + t\right) \left(t - y\right) \left(t - z\right)}}{2 \left(t - z\right)} = \lambda \\
- \frac{\sqrt{s \left(- d + s\right) \left(s - w\right) \left(s - x\right)}}{2 \left(s - w\right)} = \lambda \\
- \frac{\sqrt{s \left(- d + s\right) \left(s - w\right) \left(s - x\right)}}{2 \left(- d + s\right)} - \frac{\sqrt{t \left(- d + t\right) \left(t - y\right) \left(t - z\right)}}{2 \left(- d + t\right)} = 0 
 \end{cases}
\end{equation}
En combinant les équations 1 et 4, on trouve que $x=w$ et en combinant les équations 2 et 3, on trouve que $y=z$.Finalement il suffit de remplacer ce résultat dans l'équation 5 pour trouver que $x=y=w=z$.\\ Nous avons ainsi trouvé que les 4 cotés doivent être égaux. Or pour, un quadrilatère régulier une telle figure pourrait être soit un losange, soit un carré.\ Ce résultat qui n'est pas unique est peut être du aux théorème des extrêmes liée qui nous fournit soit une valeur minimale soit maximale de la fonction étudié ( Ce qui assez compréhensible dans le cas d'un losange).\\
Cependant, le choix n'est pas difficile.\ Regardons ce losange ($x=y=z=w$) :
\begin{center}
\begin{tikzpicture}[scale=1.25]%,cap=round,>=latex]
\coordinate [label=left:$A$] (A) at (0cm,0cm);
\coordinate [label=right:$B$] (B) at (2cm,.0cm);
\coordinate [label=above:$C$] (C) at (3cm,1.73cm);
\coordinate [label=above:$D$] (D) at (1cm,1.73cm);
\draw (A) -- node[below] {$x$} (B) -- node[right] {$y$}(C) -- node[above] {$z$} (D)--node[left] {$w$}(A);
\draw (A)--node[right]{$d$}(C);
\draw (D)--(B);
\draw (1.38, 1.07)--(1.59, 1.19)--(1.7, 0.99);
\end{tikzpicture}
\end{center}

Pour calculer sa surface, il suffit de le diviser en deux triangles DAB et  DCB qui ont la même surface.\ L'aire totale est donc  $S = \frac{1}{2}.2. x^{2}sin(\widehat{DAB})= x^{2}sin(\widehat{DAB})$.\\
Il est facile de remarque que la valeur maximale de ce losange est telle que $sin(\widehat{DAB})=1$.\ Ce qui veut dire que $\widehat{DAB}=\frac{\pi}{2}$.\\
Nous pouvant ainsi en déduire que le quadrilatère qui, à un périmètre fixé, a une surface maximale est un \textbf{carré} dont les cotés sont égaux à $\frac{P}{4}$.
\chapter{Un cas moins trivial}
\section{Question 1}
Afin d'éviter quelques hypothèses faites dans la partie précédente (raisonnement sur les angles),\ on peut utiliser le fait que tout quadrilatère à périmètre fixé et qui a une surface maximale, est nécessairement cyclique. \ C'est à dire que ses sommets appartiennent au même cercle.\\
Ainsi, on pourra utiliser \textbf{la formule de Brahmagupta}:
\begin{equation}
   K={\sqrt{(s-a)(s-b)(s-c)(s-d)}} 
\end{equation}
Avec K l'aire du quadrilatère et $a,b,c$ et $d$ ses cotés.\ $s$ représente le demi-périmètre qui est égal à $\frac{a+b+c+d}{2}$.\\
\indent Soit $f:R^{4}\longrightarrow$ R une fonction de classe $C^{1}$ , $g:R^{4}\longrightarrow$R une fonction de classe
$C^{1}$ et soit $ G = \lbrace (x_{1},...,x_{n})\vert g(x_{1},...,x_{n})=0\rbrace.$\\
\indent Posons 
\begin{equation}
    f(a,b,c,d)={\sqrt{(s-a)(s-b)(s-c)(s-d)}} 
\end{equation}

Dans ce cas la, on a deux contraintes à savoir que le périmètre doit être fixé et un coté du quadrilatère doit être égal à 1.\\
\indent Soit $g_{1}$ et $g_{2}$ réspectivement la première et la deuxième contrainte, telle que:
\begin{align*}
     g_{1}(a,b,c,d)=a+b+c+d-2s,\quad  & g_{2}(a,b,c,d)=c-1
\end{align*}
\part{En dimension infinie}






\end{document}
