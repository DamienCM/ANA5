

%----------------------------------------------------------------------------------------
%	PACKAGES AND OTHER DOCUMENT CONFIGURATIONS
%----------------------------------------------------------------------------------------

\documentclass[12pt]{report}
\usepackage[french]{babel}
\usepackage[utf8]{inputenc} % Required for inputting international characters
\usepackage[T1]{fontenc} % Output font encoding for international characters
\usepackage{mathpazo} % Palatino font
\usepackage{fancyhdr}
\usepackage{hyperref}
\usepackage{amsmath}
\usepackage{amsfonts}
\usepackage{amssymb}

\addto\captionsfrench{\renewcommand{\chaptername}{Cas}}

\begin{document}

%----------------------------------------------------------------------------------------
%	TITLE PAGE
%----------------------------------------------------------------------------------------

\begin{titlepage} % Suppresses displaying the page number on the title page and the subsequent page counts as page 1
	\newcommand{\HRule}{\rule{\linewidth}{0.5mm}} % Defines a new command for horizontal lines, change thickness here
	
	\center % Centre everything on the page
	
	%------------------------------------------------
	%	Headings
	%------------------------------------------------
	
	\textsc{\LARGE INSA de Strasbourg}\\[1.2cm] % Main heading such as the name of your university/college
	
	\textsc{\Large Projet Mathématique}\\[0.5cm] % Major heading such as course name
	
	
	%------------------------------------------------
	%	Title
	%------------------------------------------------
	
	\HRule\\[0.4cm]
	
	{\huge\bfseries Problème isopérimétrique}\\[0.4cm] % Title of your document
	
	\HRule\\[1.5cm]
	
	%------------------------------------------------
	%	Author(s)
	%------------------------------------------------
	
	\begin{minipage}{0.4\textwidth}
		\begin{flushleft}
			\large
			\textit{Auteurs}\\
			\textsc{Abrini} Mouad % Your name
			\textsc{Cartier Millon} Damien % Your name
		\end{flushleft}
	\end{minipage}
	~
	\begin{minipage}{0.4\textwidth}
		\begin{flushright}
			\large
			\textit{Encadrant}\\
			M. Jean Romain \textsc{Heu} % Supervisor's name
		\end{flushright}
	\end{minipage}
	
	% If you don't want a supervisor, uncomment the two lines below and comment the code above
	%{\large\textit{Author}}\\
	%John \textsc{Smith} % Your name
	
	%------------------------------------------------
	%	Date
	%------------------------------------------------
	
	\vfill\vfill\vfill % Position the date 3/4 down the remaining page
	
	{\large\today} % Date, change the \today to a set date if you want to be precise
	
	%------------------------------------------------
	%	Logo
	%------------------------------------------------
	
	%\vfill\vfill
	%\includegraphics[width=0.2\textwidth]{placeholder.jpg}\\[1cm] % Include a department/university logo - this will require the graphicx package
	 
	%----------------------------------------------------------------------------------------
	
	\vfill % Push the date up 1/4 of the remaining page
	
\end{titlepage}

%----------------------------------------------------------------------------------------
\tableofcontents
\part{Fonctions de plusieurs variables}

\chapter{Cas du triangle}
\section{Question 1}
Comme indiqué sur la question, on peut utiliser la formule de Héron pour calculer l'aire d'un triangle.
Pour ce faire, il suffit d'avoir le périmètre du triangle.\\
\indent Soit P le périmètre et A l'aire d'un triangle dont les côtés ont pour mesures a, b et c. Posons alors\\
\begin{equation}
   g(a,b,c) = a + b + c - P
\end{equation}
Posons $s = \frac{P}{2}$ le demi-périmètre qui sera fixé.\\ \\
La formule de Héron nous affirme que :
\begin{equation}
   f(a,b,c) = A(a,b,c) = \sqrt[2]{s(s-a)(s-b)(s-c)}
\end{equation}
\clearpage
\section{Question 2}
Nous devons ainsi calculer le gradient de g et f.

\begin{align*}
\overrightarrow{\nabla}f(a,b,c) = 
\left(\begin{matrix}
\frac{\partial f(a,b,c)}{\partial a} \\ \\
\frac{\partial f(a,b,c)}{\partial b} \\ \\
\frac{\partial f(a,b,c)}{\partial c} \\ \\
\end{matrix}\right),\quad          &  \overrightarrow{\nabla}g(a,b,c) = 
\left(\begin{matrix}
\frac{\partial g(a,b,c)}{\partial a} \\ \\
\frac{\partial g(a,b,c)}{\partial b} \\ \\
\frac{\partial g(a,b,c)}{\partial c} \\ \\
\end{matrix}\right)      
\end{align*}
\indent Ce qui nous donne (en utilisant une fonction Python) :
\begin{align*}
\overrightarrow{\nabla}f(a,b,c) = 
\left(\begin{matrix}
- \frac{\sqrt{s \left(- a + s\right) \left(- b + s\right) \left(- c + s\right)}}{2 \left(- a + s\right)}\\ \\
- \frac{\sqrt{s \left(- a + s\right) \left(- b + s\right) \left(- c + s\right)}}{2 \left(- b + s\right)}\\ \\
- \frac{\sqrt{s \left(- a + s\right) \left(- b + s\right) \left(- c + s\right)}}{2 \left(- c + s\right)} \\ \\
\end{matrix}\right),\quad          &  \overrightarrow{\nabla}g(a,b,c) = 
\left(\begin{matrix}
1 \\ 
1 \\ 
1 \\ 
\end{matrix}\right)      
\end{align*}
\indent Le théorème des extremas liées nous affirme que : 
\begin{equation}
   \overrightarrow{\nabla}f = \lambda \overrightarrow{\nabla}g
\end{equation}
Ce qui nous donne un système :
$\begin{cases} - \frac{\sqrt{s \left(- a + s\right) \left(- b + s\right) \left(- c + s\right)}}{2 \left(- a + s\right)} = \lambda \\
- \frac{\sqrt{s \left(- a + s\right) \left(- b + s\right) \left(- c + s\right)}}{2 \left(- b + s\right)} = \lambda \\
- \frac{\sqrt{s \left(- a + s\right) \left(- b + s\right) \left(- c + s\right)}}{2 \left(- c + s\right)} = \lambda \end{cases}$











\part{En dimension infinie}






\end{document}
